%%%%%%%%%%%%%%%%%%%%%%%%%%%%%%%%%%%%%%%%%%%%%%%%%%%%%%%%%%%%%%%%%%%%%%
% Template modified by Fernando Paolo from:
%
% LaTeX Template: Designer's CV
%
% Source: http://www.howtotex.com
% 
% Feel free to distribute this example, but please keep the referral
% to HowToTeX.com
% 
% Date: March 2012
%
% Modified by Lim Lian Tze to support multiple pages using fix provided at
% http://www.howtotex.com/templates/creating-a-designers-cv-in-latex/
% Date: November 2014
%
% Modified by Fernando Paolo to fit more information by changing the
% page layout (white space and column sizes).
% Data: September 2015
%
% To compile simply run:
%
% pdflatex source.tex
%%%%%%%%%%%%%%%%%%%%%%%%%%%%%%%%%%%%%%%%%%%%%%%%%%%%%%%%%%%%%%%%%%%%%%

%%%%%%%%%%%%%%%%%%%%%%%%%%%%%%%%%%%%%
% Document properties and packages
%%%%%%%%%%%%%%%%%%%%%%%%%%%%%%%%%%%%%
\documentclass[a4paper,11pt,final]{memoir}

% misc
\renewcommand{\familydefault}{bch}  % font
\pagestyle{empty}                   % no pagenumbering
\setlength{\parindent}{0pt}         % no paragraph indentation

% required packages (add your own)
\usepackage{flowfram}                                       % column layout
\usepackage[top=1cm,left=1cm,right=1cm,bottom=1cm]{geometry}% margins
\usepackage{graphicx}                                       % figures
\usepackage{url}                                            % URLs
\usepackage[usenames,dvipsnames]{xcolor}                    % color
\usepackage{multicol}                                       % columns env.
    \setlength{\multicolsep}{0pt}
\usepackage{paralist}                                       % compact lists
\usepackage{tikz}
\usepackage[nodayofweek]{datetime}
\usepackage{hyperref}

% customize date
\newdateformat{mydate}{\twodigit{\THEDAY}{ }\shortmonthname[\THEMONTH], \THEYEAR}

%%%%%%%%%%%%%%%%%%%%%%%%%%%%%%%%%%%%%
% Create column layout
%%%%%%%%%%%%%%%%%%%%%%%%%%%%%%%%%%%%%
% define length commands
\setlength{\vcolumnsep}{\baselineskip}
\setlength{\columnsep}{\vcolumnsep}

% left frame
\newflowframe{0.16\textwidth}{\textheight}{0pt}{0pt}[left]
    \newlength{\LeftMainSep}
    \setlength{\LeftMainSep}{0.16\textwidth}
    \addtolength{\LeftMainSep}{0.75\columnsep}
 
% small static frame for the vertical line
\newstaticframe{1.5pt}{\textheight}{\LeftMainSep}{0pt}
 
% content of the static frame
\begin{staticcontents}{1}
\hfill
\tikz{%
    \draw[loosely dotted,color=NavyBlue,line width=1.5pt,yshift=0]
    (0,0) -- (0,\textheight);}
\hfill\mbox{}
\end{staticcontents}
 
% right frame
\addtolength{\LeftMainSep}{1.5pt}
\addtolength{\LeftMainSep}{1\columnsep}
\newflowframe{0.75\textwidth}{\textheight}{\LeftMainSep}{0pt}[main01]


%%%%%%%%%%%%%%%%%%%%%%%%%%%%%%%%%%%%%
% define macros (for convience)
%%%%%%%%%%%%%%%%%%%%%%%%%%%%%%%%%%%%%
\newcommand{\Sep}{\vspace{1.25em}}
\newcommand{\SmallSep}{\vspace{0.5em}}

\newenvironment{Research}
    {\ignorespaces\textbf{\color{NavyBlue} Research}}
    
\newcommand{\CVSection}[1]
    {\Large\textbf{#1}\par
    \SmallSep\normalsize\normalfont}

\newcommand{\CVItem}[1]
    {\textbf{\color{NavyBlue} #1}}

%%%%%%%%%%%%%%%%%%%%%%%%%%%%%%%%%%%%%
% Begin document
%%%%%%%%%%%%%%%%%%%%%%%%%%%%%%%%%%%%%
\begin{document}

% Left frame
%%%%%%%%%%%%%%%%%%%%
%
% Upload your own photo using the files menu
%\begin{figure}
%    \hfill
%    \includegraphics[width=0.6\columnwidth]{cv-photo.png}
%    \vspace{-7cm}
%\end{figure}

\begin{flushright}\small
    crocha@ucsd.edu\\[.1cm]
    crocha700.github.io\\[.1cm]
    github: crocha700\\[.1cm]
    vimeo: crocha700\\[.1cm]
    ORCID: 0000-0003-4063-5468\\[.1cm] 
    \textcolor[gray]{0.45}{\mydate\today}\\[.1cm]
\end{flushright}\normalsize
\framebreak

% Right frame
%%%%%%%%%%%%%%%%%%%%
\Huge\bfseries { \color{NavyBlue}  Cesar B Rocha} \\
\Large\bfseries Physical oceanographer \\

\normalsize\normalfont

%%% About me
\begin{Research}
Physical oceanography, particularly the dynamics of unsteady
mesoscale to submesocale flows, and the Brazil Current System. 
Geophysical fluid dynamics, particularly stirring and mixing in geophysical turbulence.  
Numerical methods, specifically Galerkin approximations.
Data science and oceanographic big data.
\end{Research}

\Sep

%%% Education
\CVSection{Education}

Current, PhD in Oceanography, University of California, San Diego \\
2013 MSc in Oceanography, University of S\~ao Paulo, Brazil \\
2011, BSc in Oceanography, University of S\~ao Paulo, Brazil

\Sep

%%% Experience
\CVSection{Experience}

\CVItem{2015, Fellow in Geophysical Fluid Dynamics, GFD Program, WHOI}\\
Coupled reduced equations for strongly stratified flows.
\SmallSep

\CVItem{2013--Current, Graduate Student Researcher, SIO/UCSD}\\
Stratified planetary turbulence and dynamics of the upper ocean.
\SmallSep

\CVItem{2012, Visiting student, University of Massachusetts Dartmouth}\\
Quasigeostrophic modes and surface quasigeostrophic solutions.
\SmallSep

\CVItem{2011--2013, Master Student, University of Sao Paulo}\\
Energetics and dynamics of the Brazil Current System.


\Sep

\CVSection{Publications}

\CVItem{In preparation}\\

%6. \textbf{Rocha, C. B.} and C. Wunsch: Inertia-gravity waves in the ECCO2 1/48$^\circ$ Global Estimate With 
%                                        Acoustic Implications, to be submitted to JPO. 
%\SmallSep

%5. \textbf{Rocha, C. B.} and W. R. Young: Oceanic effective diffusivity: geographic variability, to be submitted to JFM. 
%\SmallSep

%4. \textbf{Rocha, C. B.} and W. R. Young: Oceanic effective diffusivity: theory, to be submitted to JFM. 
%\SmallSep

%3. \textbf{Rocha, C. B.} and W. R. Young: Are oceanic mesoscales to submesoscales chaotic advective?, to be submitted to GRL. 
%\SmallSep

%3. \textbf{Rocha, C. B.} and W. R. Young: A new regime of Homogeneous Quasi-geostrophic Turbulence, to be submitted to JFM. 
%\SmallSep

2. Ardhuin, F., Rascle, N., Chapron, B., Gula, J., Molemaker, J., Gille, S., Menemenlis, D., \textbf{Rocha, C. B.}: Small scale currents have large effects on dominant ocean waves, to be submitted to GRL.

\SmallSep

1. Chereskin, T. K., Gille, S. T., \textbf{Rocha, C. B.}: Wave-vortex decomposition in the Southern California Current System, to be submitted to JPO. 

\SmallSep

\CVItem{Peer-reviewed}\\

4.\,\textbf{Rocha, C. B.};  Chereskin, T. K.; Gille, S. T. and Menemenlis, D., 2016: ``Mesoscale to submesoscale wavenumber spectra in Drake Passage'', \textit{J. Phys. Oceanogr.}, 46 (2), 601-620, doi:10.1175/JPO-D-15-0087.1. 

\SmallSep

3.\,\textbf{Rocha, C. B.};  Young, W. R. and Grooms, I., 2016: ``On Galerkin approximations of the surface-active quasi-geostrophic equations'', \textit{J. Phys. Oceanogr.}, 46 (1), 125-139, \texttt{doi:10.1175/JPO-D-15-0073.1} 

\SmallSep

2.\,\textbf{Rocha, C. B.};  da Silveira, I. C. A., Castro, B. ,M. and Lima, J. A. M., \textbf{2014}: ``Vertical structure, energetics and dynamics of the Brazil Current System at 22$^\circ$S-28$^\circ$S'', \textit{ J. Geophys. Res., 119,} \href{http://onlinelibrary.wiley.com/doi/10.1002/2013JC009143/abstract}{\texttt{doi:10.1002/2013JC009143}.} 

\SmallSep

1.\,\textbf{Rocha, C. B.}; Tandon, A.; da Silveira, I. C. A. and Lima, J. A. M., \textbf{2013}: ``Traditional Quasi-geostrophic modes and Surface Quasi-geostrophic solutions in the Southwestern Atlantic'', \textit{ J. Geophys. Res., 118 (5),} \href{http://dx.doi.org/10.1002/jgrc.20214}{\texttt{doi:10.1002/jgrc.20214}.} 

\SmallSep


\CVItem{Grey literature}\\

1.\, \textbf{Rocha, C. B.}, 2015: Coupled reduced equations for strongly stratified flows,  Proceedings of the Geophysical Fluid Dynamics Program,  Woods Hole Oceanographic Institution, Woods Hole, MA.


\clearpage
\framebreak
\framebreak

\CVSection{Invited Seminars}

1. Oceans and Cryosphere Seminar Series, Jet Propulsion Laboratory, Fall 2015

\Sep

\CVSection{Software}

2.\, Core developer for ``Python quasigeostrophic model'' (PyQG),

\href{http://dx.doi.org/10.5281/zenodo.30517}{\texttt{doi.org/10.5281/zenodo.30517}} 

\SmallSep

1.\, Core developer for ``Spectral Analysis in Python'' (PySpec),

\href{http://dx.doi.org/10.5281/zenodo.31596}{\texttt{doi.org/10.5281/zenodo.31596}} 

\Sep


\CVSection{Service}


2016, Reviewer for DSR-I, GRL. \\

2016, Representative of physical oceanography curricular group students in the student committee for
faculty search in ``large scale observational physical oceanography''. The student  committee 
reports to the main faculty search committee.\\

2015-2016, Mentor for 1st yr SIO Graduate Students. \\

2015, Representative of physical oceanography curricular group students in the committee for
the Scripps Institution of Oceanography teaching award.




%2015, Reviewer for JGR-Oceans, JPO, GRL, Nature, Science, NSF, NASA

\Sep

\CVSection{Honors \& Awards}
2016, NASA Earth \& Space Science Graduate Fellowship\\
2015, Geophysical Fluid Dynamics Fellowhsip, Woods Hole Oceanographic Institution\\
2011, Best BSc thesis in Oceanography, University of Sao Paulo

\Sep

%%% Skills
\CVSection{Skills}

\CVItem{Programming}\\
Python , C, Fortran 90, Shell-Script, Matlab, git, mercurial, markdown

\SmallSep

\CVItem{Languages}\\
English (fluent), Portuguese (native), Spanish (professional fluency)

\Sep

%%% Skills
\CVSection{Membership}

%Golden Key International Honour Society, American Geophysical Union, The Oceanography Society, NumFOCUS (open code, better science)
 American Geophysical Union, The Oceanography Society, NumFOCUS


\Sep

%%% other
\CVSection{Other interests}

Data science, Scientific reproducibility, Free software, Open science, History and philosophy of science,
 Bossa Nova, Portuguese literature 

\Sep

\CVSection{References}

\CVItem{William R. Young}\\
Distinguished Professor of Oceanography, UC San Diego, \texttt{wryoung@ucsd.edu}

\SmallSep

\CVItem{Sarah T. Gille}\\
Professor of Oceanography, UC San Diego, \texttt{sgille@ucsd.edu}

\SmallSep

\CVItem{Teresa K. Chereskin}\\
Research Oceanographer and Senior Lecturer, UC San Diego, \texttt{tchereskin@ucsd.edu}

\SmallSep

\CVItem{Myrl Hendershott}\\
Professor of Oceanography, UC San Diego, \texttt{mch@ucsd.edu}


%\SmallSep

%\CVItem{Dimitris Menemenlis}\\
%Research Scientist, JPL/NASA, \texttt{dimitris.menemenlis@jpl.nasa.gov}








%% You'll need these 3 lines at the end of each page!
%\clearpage
%\framebreak
%\framebreak
%
%(\emph{Honors \& Awards continued})\\
%2007--2008, Brazilian Ministry of Sci. \& Tech. Fellowship (Masters)\\
%2007, Honor Mention (2nd best BSc Thesis), University of S\~ao Paulo\\
%2005--2006, S\~ao Paulo Research Foundation Fellowship (Undergrad)\\
%2004, Brazilian Ministry of Sci. \& Tech. Fellowship (Undergrad)
%
%\Sep
%
%\CVSection{Teaching}
%Aug--Dec 2008, Computing for Geophysicists, T.A. at University of S\~ao Paulo\\
%Mar--Jul 2008, Introduction to Geophysics, T.A. at University of S\~ao Paulo
%
%\Sep
%
%\CVSection{Fieldwork}
%Dec 2004, Geophysical and Geological Survey, SE Brazil (12 days at sea)\\
%Jan 2004, Brazilian Antarctic Research Station (30 days in Antarctica)\\
%Dec 2003, Environmental Monitoring Program, NE Brazil (14 days at sea)\\
%Jan 2003, Oceanographic Moorings II, SE Brazil (8 days at sea)\\
%Jul 2002, Oceanographic Moorings I, SE Brazil (9 days at sea)




%% You'll need these 3 lines at the end of each page!
%\clearpage
%\framebreak
%\framebreak
%
%\CVItem{2005 - 2007, Vivamus vel bibendum}\\
%Proin rutrum pharetra molestie. Cras sollicitudin nulla nec leo lobortis in tristique purus pretium. Ut eu felis nulla.
%\Sep
%
%
%% Skills
%\CVSection{Skills}
%\CVItem{Platforms}
%\begin{multicols}{3}
%\begin{compactitem}[\color{NavyBlue}$\circ$]
%    \item Lorem 
%    \item Ipsum 
%\end{compactitem}
%\end{multicols}
%\SmallSep
%
%\CVItem{Computer software}
%\begin{multicols}{3}
%\begin{compactitem}[\color{NavyBlue}$\circ$]
%    \item Lorem 
%    \item Ipsum 
%    \item Dolor 
%    \item Sit 
%    \item Amet
%    \item Consectetur 
%    \item Adipiscing 
%    \item Elit
%    \item \ldots
%\end{compactitem}
%\end{multicols}
%\Sep 
%
%% References
%\CVSection{References}
%References upon request.

%%%%%%%%%%%%%%%%%%%%%%%%%%%%%%%%%%%%%
% End document
%%%%%%%%%%%%%%%%%%%%%%%%%%%%%%%%%%%%%
\end{document}

