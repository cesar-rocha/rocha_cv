%%%%%%%%%%%%%%%%%%%%%%%%%%%%%%%%%%%%%%%%%
% Plasmati Graduate CV
% LaTeX Template
% Version 1.0 (24/3/13)
%
% This template has been downloaded from:
% http://www.LaTeXTemplates.com
%
% Original author:
% Alessandro Plasmati (alessandro.plasmati@gmail.com)
%
% License:
% CC BY-NC-SA 3.0 (http://creativecommons.org/licenses/by-nc-sa/3.0/)
%
% Important note:
% This template needs to be compiled with XeLaTeX.
% The main document font is called Fontin and can be downloaded for free
% from here: http://www.exljbris.com/fontin.html
%
%%%%%%%%%%%%%%%%%%%%%%%%%%%%%%%%%%%%%%%%%

%----------------------------------------------------------------------------------------
%	PACKAGES AND OTHER DOCUMENT CONFIGURATIONS
%----------------------------------------------------------------------------------------

% Modified to be compiled with pdfLaTeX
% CROCHA, Fall 2013

\documentclass[pagestart=firstchapter]{article}
\usepackage[USenglish,english,american]{babel}

\usepackage[utf8]{inputenc}
\usepackage[T1]{fontenc}
\usepackage{geometry}
\geometry{verbose,a4paper,tmargin=3cm,bmargin=2cm,lmargin=1.5cm,rmargin=2.cm}
\usepackage{longtable}
\usepackage{subfigure}
\usepackage{float}
\usepackage{graphicx}
\usepackage{setspace}
\onehalfspacing
\usepackage{amsmath,amsthm}
\usepackage{amsfonts}
\usepackage{amssymb}
\usepackage{esint}
\usepackage{color}
\usepackage{multicol}
\usepackage{hyperref}


\begin{document}

\pagestyle{empty} % Removes page numbering


%----------------------------------------------------------------------------------------
%	NAME AND CONTACT INFORMATION
%----------------------------------------------------------------------------------------

\par{\centering{\Huge \textbf{Cesar Barbedo Rocha}}\bigskip\par} % Your name

\begin{center}
    \begin{tabular}{rl}
        Graduate Student Researcher ~~~~ &  ~~~~~~~~~~ 9500 Gilman Drive, M/C 0213 \\ 
        Climate-Ocean-Atmosphere Program~  &  ~~~~~~~~~~ La Jolla, CA 92093 \\
       Scripps Institution of Oceanography~    & ~~~~~~~~~~ \href{mailto:crocha@ucsd.edu}{\textcolor{blue}{crocha@ucsd.edu}}\\
    University of California, San Diego ~~~& ~~~~~~~~~~~\url{http://www-pord.ucsd.edu/~crocha}\\
    \end{tabular}
\end{center}

%----------------------------------------------------------------------------------------
% PERSONAL INFORMATION	
%----------------------------------------------------------------------------------------

%------------------------------------------------



%----------------------------------------------------------------------------------------
%	EDUCATION
%----------------------------------------------------------------------------------------

\section*{Education}

\begin{tabular}{rl}	

\textsc{Current}  & PhD student in \textsc{Oceanography}, \textbf{Scripps Institution of Oceanography,}\\
                  & \textbf{University of California, San Diego.}\\
                  & \small Advisor: William R. Young.\\
&\\
\textsc{2013} & Bachelor (2011) and Master (2013) of Science in \textsc{Oceanography}, \textbf{University of S\~ao Paulo}, Brazil;\\
                      &  with exchange period at \textbf{University of Massachusetts Dartmouth}.\\

                      & Theses: ``On the Brazil Current recirculation cell: A theoretical parametric approach'' (BS);\\
                      & ``\href{http://www.teses.usp.br/teses/disponiveis/21/21135/tde-26092013-185418/en.php}{Energetics, Baroclinic Instability and the Vertical Structure of the Brazil Current at $22^\circ$--$28^\circ$S}'' (MS) \\
& \small Advisors: Ilson Carlos A.  \textsc{da Silveira} (IOUSP) and Amit \textsc{Tandon} (UMass Dartmouth) \\
&\\
%------------------------------------------------

\end{tabular}

%----------------------------------------------------------------------------------------

%----------------------------------------------------------------------------------------
%	RESEARCH INTERESTS
%----------------------------------------------------------------------------------------

\section*{Research Statement}
I use a mix of theory, analysis of observations, and computational simulations to understand how the ocean works. In particular, my main current research interest is on the physics of the upper ocean, specifically the eddy dynamics at meso to submesoscales.
%----------------------------------------------------------------------------------------
%	PUBLICATIONS
%----------------------------------------------------------------------------------------

\section*{Publications}

\begin{tabular}{rl}
\textsc{Coming up soon} &  \\
\end{tabular}

\begin{itemize}
    \item \textbf{Rocha, C. B.};  Chereskin, T. K.; Gille, S. T. and Menemenlis, D., {\textit{in prep.} for JPO}: ``Drake Passage kinetic energy and sea-surface height horizontal wavenumber spectra in the mesoscale to submesoscale range (10-200 km)''. 
\end{itemize}

\begin{itemize}
    \item \textbf{Rocha, C. B.};  Young, W. R. and Grooms, I., {\textit{ in prep.} for JPO}: ``On Galerkin approximations for the quasi-geostrophic equations''. 
\end{itemize}



\begin{tabular}{rl}
\hspace{-.55cm}\textsc{Journal Articles} &  \\
\end{tabular}

\begin{itemize}
    \item \textbf{Rocha, C. B.};  da Silveira, I. C. A., Castro, B. ,M. and Lima, J. A. M., \textbf{2014}: ``Vertical structure, energetics and dynamics of the Brazil Current System at 22$^\circ$S-28$^\circ$S'', \textit{\bf J. Geophys. Res., 119,} \href{http://onlinelibrary.wiley.com/doi/10.1002/2013JC009143/abstract}{\texttt{doi: 10.1002/2013JC009143}.} 

    \item \textbf{Rocha, C. B.}; Tandon, A.; da Silveira, I. C. A. and Lima, J. A. M., \textbf{2013}: ``Traditional Quasi-geostrophic modes and Surface Quasi-geostrophic solutions in the Southwestern Atlantic'', \textit{\bf J. Geophys. Res., 118 (5),} \href{http://dx.doi.org/10.1002/jgrc.20214}{\texttt{doi:10.1002/jgrc.20214}.} 
\end{itemize}

\begin{tabular}{rl}
\hspace{-.55cm}\textsc{Technical notes} ~~~~~~~~~~~~~ &  \\
\end{tabular}

\begin{itemize}
    \item   Piola, A. R. and \textbf{Rocha, C. B.}, \textbf{2010}: ``Recent CTD/XBT comparisons in the Brazil Current''. Prepared for  \textit{Second XBT Fall Rate Workshop}, August 25-27, 2010, Hamburg, Germany. 
\end{itemize}




%----------------------------------------------------------------------------------------
%   INVITED TALKS	
%----------------------------------------------------------------------------------------


\section*{Invited Talks}

\begin{tabular}{rl}
    \textsc{July 2013} & \textbf{Energetics and Dynamics of the Brazil Current System at 22$^\circ$S-28$^\circ$S}\\
    & Seminar at FURG, Rio Grande, Brazil. Host: Prof. Paulo H. R. Calil. \\
\end{tabular}


%----------------------------------------------------------------------------------------
%   FIELD EXPERIENCE	
%----------------------------------------------------------------------------------------


\section*{Field Experience}
During the cruises listed below, I actively participated in instrument operation (XBT, CTD, ADCP, LADCP), near-real time data processing, and adaptive sampling. I co-authored the reports for the CERES III and IV cruises.\\

\begin{tabular}{rl}

    \textsc{July  2010} & \href{http://www.aoml.noaa.gov/phod/research/moc/samoc/}{``SAMOC Experiment''} - SAM03 PD 2010-05 cruise  aboard R/V "Puerto Deseado" (CONICET) \\
                        & Southwestern Atlantic (35$^\circ$S); 13 days at sea. Chief scientist: Alberto Piola.\\ 
  &\\ 
 \textsc{June  2010} & ``CERES Experiment'' - CERES IV cruise  aboard R/V ``Antares'' (Brazilian Navy)  \\
                     & Southwestern Atlantic (22$^\circ$-30$^\circ$S); 15 days at sea. Chief scientist: Wellington Ceccopieri\\
  &\\ 
\textsc{2008 -- 2009} & ``CERES Experiment'' - CERES I, II, and III cruises  aboard R/V ``Gyre''  \\
                     & Southwestern Atlantic (22$^\circ$-26$^\circ$S); $\sim$ 8 days at sea. Chief scientist: Wellington Ceccopieri\\
  &\\ 

\end{tabular}


%----------------------------------------------------------------------------------------
%   TEACHING EXPERIENCE	
%----------------------------------------------------------------------------------------


\section*{Teaching Experience}

\begin{tabular}{rl}
    \textsc{Spring 2011} & Teaching Assistant for ``Dynamical Oceanography II'' (Undergraduate)\\
    & University of S\~ao Paulo, Brazil. Instructor: Prof. Ilson Carlos A. da Silveira. \\
\end{tabular}
 
%----------------------------------------------------------------------------------------
%	AWARDS
%----------------------------------------------------------------------------------------

\section*{Awards, Fellowships, and Honors}
\begin{tabular}{rl}
\textsc{2015} & ``Geophysical Fluid Dynamics Fellow, Woods Whole Oceanographic Institution''\\
\textsc{2011} & ``Best BS thesis, Oceanographic Institute, University of S\~ao Paulo, Brazil''
\end{tabular}




%----------------------------------------------------------------------------------------
%	COMPUTER SKILLS 
%----------------------------------------------------------------------------------------

\section*{Computer Skills}

\begin{tabular}{rl}
    \textsc{Programming}: & Python, Fortran 90, C, Shell script (bash)\\
    \textsc{Version control}: & git, mercurial\\
    \textsc{Diagramming}: & \LaTeX\\
    \textsc{Applications}: & Matlab, GMT, Ferret, \textsc{HTML} (basics)\\
    \textsc{Operating Systems}: & GNU/Linux, MacOS\\
    \textsc{PE models}: &ROMS (basics)    
\end{tabular}


%----------------------------------------------------------------------------------------
%	LANGUAGES
%----------------------------------------------------------------------------------------

\section*{Languages}

\textsc{Portuguese} (Mother tongue), \textsc{English} (Fluent), \textsc{Spanish} (Professional fluency)
%----------------------------------------------------------------------------------------
%	MEMBERSHIP
%----------------------------------------------------------------------------------------

\section*{Membership}

American Geophysical Union, The Oceanography Society, NumFOCUS (open code, better science).

%----------------------------------------------------------------------------------------
% OTHER	INTERESTS AND ACTIVITIES
%----------------------------------------------------------------------------------------

%\section*{References}
%
%William R. Young, Professor of Physical Oceanography (SIO, UCSD) <\href{mailto:wryoung@ucsd.edu}{wryoung@ucsd.edu}>\\
%Sarah T. Gille, Professor of Physical Oceanography (SIO, UCSD) <\href{mailto:sgille@ucsd.edu}{sgille@ucsd.edu}>\\
%Teresa K. Chereskin, Research Oceanographer and Senior Lecturer (SIO, UCSD) <\href{mailto:tchereskin@ucsd.edu}{tchereskin@ucsd.edu}>\\
%Amit Tandon, Professor of Physics and Oceanography (UMassD) <\href{mailto:atandon@umassd.edu}{atandon@umassd.edu}> 
%

%----------------------------------------------------------------------------------------
\end{document}
