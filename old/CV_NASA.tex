%%%%%%%%%%%%%%%%%%%%%%%%%%%%%%%%%%%%%%%%%%%%%%%%%%%%%%%%%%%%%%%%%%%%%%%%
%%%%%%%%%%%%%%%%%%%%%% Simple LaTeX CV Template %%%%%%%%%%%%%%%%%%%%%%%%
%%%%%%%%%%%%%%%%%%%%%%%%%%%%%%%%%%%%%%%%%%%%%%%%%%%%%%%%%%%%%%%%%%%%%%%%

%%%%%%%%%%%%%%%%%%%%%%%%%%%%%%%%%%%%%%%%%%%%%%%%%%%%%%%%%%%%%%%%%%%%%%%%
%% NOTE: If you find that it says                                     %%
%%                                                                    %%
%%                           1 of ??                                  %%
%%                                                                    %%
%% at the bottom of your first page, this means that the AUX file     %%
%% was not available when you ran LaTeX on this source. Simply RERUN  %%
%% LaTeX to get the ``??'' replaced with the number of the last page  %%
%% of the document. The AUX file will be generated on the first run   %%
%% of LaTeX and used on the second run to fill in all of the          %%
%% references.                                                        %%
%%%%%%%%%%%%%%%%%%%%%%%%%%%%%%%%%%%%%%%%%%%%%%%%%%%%%%%%%%%%%%%%%%%%%%%%

%%%%%%%%%%%%%%%%%%%%%%%%%%%% Document Setup %%%%%%%%%%%%%%%%%%%%%%%%%%%%

% Don't like 10pt? Try 11pt or 12pt
\documentclass[10pt]{article}

% This is a helpful package that puts math inside length specifications
\usepackage{calc}

% Simpler bibsection for CV sections
% (thanks to natbib for inspiration)
\makeatletter
\newlength{\bibhang}
\setlength{\bibhang}{1em}
\newlength{\bibsep}
 {\@listi \global\bibsep\itemsep \global\advance\bibsep by\parsep}
\newenvironment{bibsection}%
        {\vspace{-\baselineskip}\begin{list}{}{%
       \setlength{\leftmargin}{\bibhang}%
       \setlength{\itemindent}{-23pt}%{-\leftmargin}%
       \setlength{\itemsep}{\bibsep}%
       \setlength{\parsep}{\z@}%
        \setlength{\partopsep}{0pt}%
        \setlength{\topsep}{0pt}}}
        {\end{list}\vspace{-.6\baselineskip}}
\makeatother

% Layout: Puts the section titles on left side of page
\reversemarginpar

%
%         PAPER SIZE, PAGE NUMBER, AND DOCUMENT LAYOUT NOTES:
%
% The next \usepackage line changes the layout for CV style section
% headings as marginal notes. It also sets up the paper size as either
% letter or A4. By default, letter was used. If A4 paper is desired,
% comment out the letterpaper lines and uncomment the a4paper lines.
%
% As you can see, the margin widths and section title widths can be
% easily adjusted.
%
% ALSO: Notice that the includefoot option can be commented OUT in order
% to put the PAGE NUMBER *IN* the bottom margin. This will make the
% effective text area larger.
%
% IF YOU WISH TO REMOVE THE ``of LASTPAGE'' next to each page number,
% see the note about the +LP and -LP lines below. Comment out the +LP
% and uncomment the -LP.
%
% IF YOU WISH TO REMOVE PAGE NUMBERS, be sure that the includefoot line
% is uncommented and ALSO uncomment the \pagestyle{empty} a few lines
% below.
%

%% Use these lines for letter-sized paper
\usepackage[paper=letterpaper,
            %includefoot, % Uncomment to put page number above margin
            marginparwidth=1.2in,     % Length of section titles
            marginparsep=.05in,       % Space between titles and text
            margin=0.65in,            % 1 inch margins
            includemp]{geometry}

%% Use these lines for A4-sized paper
%\usepackage[paper=a4paper,
%            %includefoot, % Uncomment to put page number above margin
%            marginparwidth=30.5mm,    % Length of section titles
%            marginparsep=1.5mm,       % Space between titles and text
%            margin=25mm,              % 25mm margins
%            includemp]{geometry}

%% More layout: Get rid of indenting throughout entire document
\setlength{\parindent}{0in}

%% This gives us fun enumeration environments. compactitem will be nice.
\usepackage{paralist}

%% Reference the last page in the page number
%
% NOTE: comment the +LP line and uncomment the -LP line to have page
%       numbers without the ``of ##'' last page reference)
%
% NOTE: uncomment the \pagestyle{empty} line to get rid of all page
%       numbers (make sure includefoot is commented out above)
%
\usepackage{fancyhdr,lastpage}
\pagestyle{fancy}
\pagestyle{empty}      % Uncomment this to get rid of page numbers
\fancyhf{}\renewcommand{\headrulewidth}{0pt}
\fancyfootoffset{\marginparsep+\marginparwidth}
\newlength{\footpageshift}
\setlength{\footpageshift}
          {0.5\textwidth+0.5\marginparsep+0.5\marginparwidth-2in}
\lfoot{\hspace{\footpageshift}%
       \parbox{4in}{\, \hfill %
                    \arabic{page} of \protect\pageref*{LastPage} % +LP
%                    \arabic{page}                               % -LP
                    \hfill \,}}
                    
% Finally, give us PDF bookmarks
\usepackage{color,hyperref}
%\definecolor{darkblue}{rgb}{0.0,0.0,0.3}
%\hypersetup{colorlinks,breaklinks,
%            linkcolor=darkblue,urlcolor=darkblue,
%            anchorcolor=darkblue,citecolor=darkblue}
\hypersetup{colorlinks,breaklinks,
            linkcolor=black,urlcolor=black,
            anchorcolor=black,citecolor=black}

%%%%%%%%%%%%%%%%%%%%%%%% End Document Setup %%%%%%%%%%%%%%%%%%%%%%%%%%%%


%%%%%%%%%%%%%%%%%%%%%%%%%%% Helper Commands %%%%%%%%%%%%%%%%%%%%%%%%%%%%

% The title (name) with a horizontal rule under it
% (optional argument typesets an object right-justified across from name
%  as well)
%
% Usage: \makeheading{name}
%        OR
%        \makeheading[right_object]{name}
%
% Place at top of document. It should be the first thing.
% If ``right_object'' is provided in the square-braced optional
% argument, it will be right justified on the same line as ``name'' at
% the top of the CV. For example:
%
%       \makeheading[\emph{Curriculum vitae}]{Your Name}
%
% will put an emphasized ``Curriculum vitae'' at the top of the document
% as a title. Likewise, a picture could be included:
%
%   \makeheading[\includegraphics[height=1.5in]{my_picutre}]{Your Name}
%
% the picture will be flush right across from the name.
\newcommand{\makeheading}[2][]%
        {\hspace*{-\marginparsep minus \marginparwidth}%
         \begin{minipage}[t]{\textwidth+\marginparwidth+\marginparsep}%
             {\large \bfseries #2 \hfill #1}\\[-0.15\baselineskip]%
                 \rule{\columnwidth}{1pt}%
         \end{minipage}}

% The section headings
%
% Usage: \section{section name}
%
% Follow this section IMMEDIATELY with the first line of the section
% text. Do not put whitespace in between. That is, do this:
%
%       \section{My Information}
%       Here is my information.
%
% and NOT this:
%
%       \section{My Information}
%
%       Here is my information.
%
% Otherwise the top of the section header will not line up with the top
% of the section. Of course, using a single comment character (%) on
% empty lines allows for the function of the first example with the
% readability of the second example.
\renewcommand{\section}[2]%
        {\pagebreak[3]\vspace{1.3\baselineskip}%
         \phantomsection\addcontentsline{toc}{section}{#1}%
         \hspace{0in}%
         \marginpar{
         \raggedright \scshape #1}#2}

% An itemize-style list with lots of space between items
\newenvironment{outerlist}[1][\enskip\textbullet]%
        {\begin{itemize}[#1]}{\end{itemize}%
         \vspace{-.6\baselineskip}}

% An environment IDENTICAL to outerlist that has better pre-list spacing
% when used as the first thing in a \section
\newenvironment{lonelist}[1][\enskip\textbullet]%
        {\vspace{-\baselineskip}\begin{list}{#1}{%
        \setlength{\partopsep}{0pt}%
        \setlength{\topsep}{0pt}}}
        {\end{list}\vspace{-.6\baselineskip}}

% An itemize-style list with little space between items
\newenvironment{innerlist}[1][\enskip\textbullet]%
        {\begin{compactitem}[#1]}{\end{compactitem}}

% An environment IDENTICAL to innerlist that has better pre-list spacing
% when used as the first thing in a \section
\newenvironment{loneinnerlist}[1][\enskip\textbullet]%
        {\vspace{-\baselineskip}\begin{compactitem}[#1]}
        {\end{compactitem}\vspace{-.6\baselineskip}}

% To add some paragraph space between lines.
% This also tells LaTeX to preferably break a page on one of these gaps
% if there is a needed pagebreak nearby.
\newcommand{\blankline}{\quad\pagebreak[3]}
\newcommand{\halfblankline}{\quad\vspace{-0.5\baselineskip}\pagebreak[3]}

% Uses hyperref to link DOI
\newcommand\doilink[1]{\href{http://dx.doi.org/#1}{#1}}
\newcommand\doi[1]{doi:\doilink{#1}}

% For \url{SOME_URL}, links SOME_URL to the url SOME_URL
%\providecommand*\url[1]{\href{#1}{#1}}
% Same as above, but pretty-prints SOME_URL in teletype fixed-width font
%\renewcommand*\url[1]{\href{#1}{\texttt{#1}}}

% For \email{ADDRESS}, links ADDRESS to the url mailto:ADDRESS
\providecommand*\email[1]{\href{mailto:#1}{#1}}
% Same as above, but pretty-prints ADDRESS in teletype fixed-width font
%\renewcommand*\email[1]{\href{mailto:#1}{\texttt{#1}}}

%\providecommand\BibTeX{{\rm B\kern-.05em{\sc i\kern-.025em b}\kern-.08em
%    T\kern-.1667em\lower.7ex\hbox{E}\kern-.125emX}}
%\providecommand\BibTeX{{\rm B\kern-.05em{\sc i\kern-.025em b}\kern-.08em
%    \TeX}}
\providecommand\BibTeX{{B\kern-.05em{\sc i\kern-.025em b}\kern-.08em
    \TeX}}
\providecommand\Matlab{\textsc{Matlab}}

\begin{document}

%%%%%%%%%%%%%%%%%%%%%%%%%%%%%%%%%%%%%%%%%%%%%%%%%%%%%%%%%%%%%%%%%%%%%%%%%%%%%%%%%%%%%%%%%%%%%%%%%%%%%%%%%%%%%%%%%%%%%%%%%%%%%%%%%%%%%%%%%%

\makeheading{\LARGE{Sarah Brody}}

%\section{Version}
%January, 2012

\section{Contact Information}
%
% NOTE: Mind where the & separators and \\ breaks are in the following
%       table.
%
% ALSO: \rcollength is the width of the right column of the table
%       (adjust it to your liking; default is 1.85in).
%
\newlength{\rcollength}\setlength{\rcollength}{2.65in}%
%
\begin{tabular}[t]{@{}p{\textwidth-\rcollength}p{\rcollength}}
Division of Earth and Ocean Sciences	     & \textit{Mobile:} (240) 988-9662 \\
Old Chemistry Building                       & \textit{Office:} (919) 681-8169 \\
Box 90227	
Duke University                              & \textit{E-mail:} \email{Sarah.Brody@duke.edu} \\
Durham, NC 27708 
\end{tabular}

\section{Education}
\textbf{Duke University}, Durham, NC \hfill \textbf{Sept. 2010 -- Present}\\
\textit{Ph.D. Candidate, Earth and Ocean Sciences}\\
\begin{loneinnerlist}
  \item Proposed dissertation title: ``Examining the physical controls on phytoplankton bloom phenology in subpolar regions.''
  \item Advisor: M. Susan Lozier \\
\end{loneinnerlist}
%
\textbf{Universisty of Pennsylvania}, Philadelphia, PA \hfill \textbf{Sept. 2006 -- May. 2010}\\
\textit{B.A. (with honors, magna cum laude), Geology}\\
\begin{loneinnerlist}
  \item Honors thesis: ``Coseismic subsidence due to great earthquakes on the Cascadia subduction zone: a geochemical proxy for relative sea level change.''
  \item Advisor: Benjamin Horton \\
\end{loneinnerlist}

\vspace{-0.1in}

\section{Research Experience}
\textbf{NOAA Geophysical Fluid Dynamic Laboratory}, Princeton, NJ \hfill \textbf{Jun. 2011 -- Aug. 2011}\\
\textit{MPOWIR Intern}\\
\begin{loneinnerlist}
  \item Worked with satellite ocean color data and the TOPAZ earth system model to compare observational and modelled phytoplankton bloom phenologies in the Souther Ocean with physical variables. 
  \item Advisor: John Dunne (climate and ecosystems group)
\end{loneinnerlist}~\\
%
\textbf{University of Pennsylvania}, Philadelphia, PA \hfill \textbf{Mar. 2008 -- Dec. 2009} \\
\textbf{Sea Level Research Laboratory}\\
\textit{Research Assistant}\\
\begin{loneinnerlist}
  \item Assisted projects related to high-resolution Holocende sea-level and paleoenvironment reconstructions by analyzing total organic carbon and grain size from marsh sediment cores.
  \item Advisor: Benjamin Horton (earth and environmental science dept.)
\end{loneinnerlist}~\\
%
\textbf{Woods Hole Oceanographic Institution}, Woods Hole, MA \hfill \textbf{May 2009 -- Aug. 2009}\\
\textit{Guest Student}\\
\begin{loneinnerlist}
  \item Assisted in fieldwork (marsh and pond coring, ground penetrating radar), analyzed sediment cores from New England salt marshes, using foraminifera to determine the relative intensities of hurricane events from the past 1,200 years.
  \item Advisor: Jeffrey Donnelly (coastal systems group)
\end{loneinnerlist}~\\
%
\textbf{Clean Air Council of Philadelphia}, Philadelphia, PA \hfill \textbf{April 2009}\\
\textit{Global Warming Policy Intern}\\
\begin{loneinnerlist}
  \item Researched proposed ammendments to the American Clean Energy and Securities Act in order to ensure that the bill was as effective as possible in mitigating the effects of climate change.

        
%\vspace{-0.25in}
\section{Publications} 
\begin{bibsection}


  \item Brody, S.R., Lozier, M.S. and Dunne, J.P. A comparison of methods to determine phytoplankton bloom initiation. (under review, \emph{Journal of Geophysical Research - Oceans})

  \item Engelhart, S.E., Horton, B.P., Vane., C.H., Nelson, A.R., Witter, R., Brody, S.R., and Hawkes, A.D.  Modern foraminifera and stable carbon isotopes of central Oregon tidal marshes and their application in paleoseismology. (under review, \emph{Paleogeography, Paleoclimatology, Paleoecology})

  
\end{bibsection}

\vspace{0.15in}

\section{Conference Abstracts} 
\begin{bibsection}
  
  \item Brody, S.R., Lozier, M.S. and Dunne, J.P. (2012). A comparison of methods to determine phytoplankton bloom initiation.  Ocean Optics Meeting, Glasgow, 8th October - 12th October 2012.

\newpage

  \item Brody, S.R., Dunne, J.P., Cassar, N. and Lozier, M.S. (2012).  Examining Phytoplankton Bloom Phenology in the Southern Ocean.  Ocean Sciences Meeting, Salt Lake City, 20th February - 24th February 2012.

  \item Engelhart, S.E., Horton, B.P., Nelson, A.R., Witter, R., Hawkes, A.D., Vane., C.H., and Brody, S.R. (2010).  Quantitative estimates of coseismic subsidence associated with the 1700 AD megathrust earthquake, Siletz Bay, Oregon.  IGCP588 Meeting, Hong Kong, 30th November - 5th December 2010
  
\end{bibsection}

\vspace{0.15in}

%\section{Teaching Assistantships}
%\textbf{EGR75: Mechanics of Solids}, Fall Semester, 2009\\
%\textbf{CE131: Matrix Structural Analysis}, Fall Semester, 2010\\
%Augmented courses by grading assignments, holding office hours, and gained teaching experience by holding class homework recitations.


\section{Honors and Awards}
\begin{loneinnerlist}
  \item Henry Darwin Rogers Award for Distincton in Geology (awarded by the Unversity of Pennsylania), May 2010
  \item University of Pennsylania Dean's List, 2009-2010
\end{loneinnerlist}

\section{Fellowships and Grants}
\begin{loneinnerlist}
  \item NASA Earth and Space Science Fellowship, May 2012 (awarded)
  \item MPOWIR NOAA internship fellowship, Jun. 2011 - Aug. 2011
  \item University of Pennsylvania College Alumni Society Undergraduate Research Grant, Dec. 2009 - May 2010.
\end{loneinnerlist}

\section{Skills}
\begin{loneinnerlist}
  \item \textit{Computer}: Microsoft Office, Matlab, Unix, Ferret, LaTex
  \item \textit{Field}: CTD/rosette package deployment and recovery
%  \item \textit{Certifications}: Fundamentals of Engineering Exam (Passed on 4/2008)
%\end{loneinnerlist}

%%%%%%%%%%%%%%%%%%%%%%%%%%%%%%%%%%%%%%%%%%%%%%%%%%%%%%%%%%%%%%%%%%%%%%%%%%%%%%%%%%%%%%%%%%%%%%%%%%%%%%%%%%%%%%%%%%%%%%%%%%%%%%%%%%%%%%%%%%

\end{document}
